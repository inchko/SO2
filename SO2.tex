\documentclass{article}
\usepackage{hyperref}
\usepackage[catalan]{babel}
\usepackage{graphicx}
\selectlanguage{catalan}

\graphicspath{ {./img} }

\hypersetup{
    colorlinks,
    citecolor=black,
    filecolor=black,
    linkcolor=black,
    urlcolor=blue
}

\title{Sistemes Operatius 2 (SO2)}
\date{2018-2019 Q2}
\author{David Soldevila}

\begin{document}

    \maketitle  

    \pagebreak
    
    \tableofcontents
    
    \pagebreak

    \section{Introducció}

    Treballarem amb x86 amb regs de 32 bits.

    Registres importants:
    \begin{itemize}
        \item EAX
        \item EBX
        \item ESI
        \item EDI
        \item ECX
        \item EBP
        \item EDX     
        \item ESP
    \end{itemize}

    \pagebreak

    \section{Mecanismes d'integritat del Sistema Operatiu}

    \subsection{Nivells de privilegi}

    Fa que els usuaris no tinguin accés al hardware. Les instruccions privilagiades només poden ser executades per l'OS. Es necessaria suport de hardware.

    \subsection{Vector d'interrupcions}

    IDT: Interrupt Descriptor Table: 256 entrades

    - 0 - 31: Exceptions
    - 32 - 47: Masked interrupts
    - 48 - 255: Software interrupts (Traps)

    El tractament de la excepcio s'hauria de fer en una sola rutina.

    \subsection{Excepcions hardware}

        S'ha de notificar al controlador de interrupcions quan s'ha acabat de tractar, per tal de poder tractar mes interrupcions.

    \subsection{Syscall Table}

        Taula on hi ha totes les funcions (addrs) amb totes les syscalls. Si només es fa servir la IDT només es pot tenir 256 syscalls.

    \subsection{sysenter}

        Permet entrar de mode sistema sense fer tot el proces.
        Per fer-ho s'ha de definir:
        
         \begin{itemize}
            \item SYSENTER\_CS\_MSR
            \item SYSENTER\_EIP\_MSR
            \item SYSENTER\_ESP\_MSR
        \end{itemize}

    \subsection{sysexit}
        Permet sortir de mode sistema sense fer tot el proces.
        Despres de restaurar tot.
        \begin{itemize}
            \item EDX $\leftarrow$ EIP
            \item ECX $\leftarrow$ ESP
        \end{itemize}
        
    \pagebreak

    \section{Creació de fitxers executables}

    \subsection{Compilar i montar}

        Un fitxer executable conte una header (informació necessaria per executar el programa, tipus de fitxer, punters als inicis de les seccions, etc.), dades inicialitzades, memoria necessaria per a la pila i dades no inicialitzades.
        
        \subsubsection{Carregador}

            Aquest programa llegiex un executable i carrega en memoria el codi i les daes i reserva espai en memoria per la pila i les dades no inicialitzades.

        \subsubsection{Espais de direccions}

            \begin{itemize}
                \item Esai de direccions logiques del processador
                \item Espai de direccions logiques del proces
                \item Espai de direccions fisiques del proces
            \end{itemize}
            
        \paragraph{Espi logic direccions logiques del proces}
        és l'espai que genera un processador quan estaa executant un proces son logiques. Per traduir les direccions fisiques a logiques es fa servir la MMU (Memory Management Unit).

        \subsubsection{MMU}

        \subparagraph{La MMU de Intel Pentium}
        per cada sement te una direccio base i una mida. I cada proces te UNA taula.
        
        A la pràctica hi ha dos segments, un per usuari i l'altre per sistema.

    \pagebreak

    \section{Gestió de porcessos}

    Un process es una unitat d'activitat que es caracteriza per la execució de uan secuenda de instruccions, un estat actual i un conjunt de recursos del sistema associats.

    \subsection{Característiques}

    \paragraph{Espai lógic de @: } imatges en memoria del proces (codi, dades i pila). I la pila del kerner.

    \paragraph{Context:} per part del hardware: valors dels registres i la tts. Per part del software: informació de planificació, dispositius...

    En cas de linux, a més a més, necessita prioritat, quantum, canals, etc.
            
    \subsection{Representació de un proces}

    Cada proces te un Process Coontrol Block (PCB)

    \begin{itemize}
        \item Identificador del proces.
        \item estat del proces.
        \item recursos del proces.
        \item Estadístiques del proces.
        \item Inforamció de planificació.
        \item Context d'execució.
            \begin{itemize}
                \item Dins el PCB.
                \item A la pila del proces i al pCB només a memòria.
            \end{itemize}
        \item etc.
    \end{itemize}

    I una pila de memòria.

    \paragraph{Per implementar-ho} 

\end{document}